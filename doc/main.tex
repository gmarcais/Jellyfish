\documentclass[english]{article}
\usepackage[latin1]{inputenc}
\usepackage{babel}
\usepackage{verbatim}

%% do we have the hyperref package?
\IfFileExists{hyperref.sty}{
   \usepackage[bookmarksopen,bookmarksnumbered]{hyperref}
}{}

%% do we have the fancyhdr package?
\IfFileExists{fancyhdr.sty}{
\usepackage[fancyhdr]{latex2man}
}{
%% do we have the fancyheadings package?
\IfFileExists{fancyheadings.sty}{
\usepackage[fancy]{latex2man}
}{
\usepackage[nofancy]{latex2man}
\message{no fancyhdr or fancyheadings package present, discard it}
}}

\newcommand{\ddash}[1]{-\,-#1}
\newcommand{\LOpt}[1]{\Opt{\ddash{#1}}}
\newcommand{\LoOpt}[1]{\oOpt{\ddash{#1}}}
\newcommand{\LOptArg}[2]{\OptArg{\ddash{#1}}{#2}}
\newcommand{\LoOptArg}[2]{\oOptArg{\ddash{#1}}{#2}}
\newcommand{\LoOptoArg}[2]{\oOptoArg{\ddash{#1}}{#2}}

\setVersion{1.1}
\setDate{2010/10/1}

\begin{document}

\begin{Name}{1}{jellyfish}{G. Marcais}{k-mer counter}{Jellyfish: A fast k-mer counter}

\Prog{Jellyfish} is a software to count $k$-mers in DNA sequences.

\end{Name}

\section{Synopsis}
\Prog{jellyfish count} \oOptArg{-o}{prefix} \oOptArg{-m}{merlength} \oOptArg{-t}{threads} \oOptArg{-s}{hashsize} \LoOpt{both-strands} \Arg{fasta} \oArg{fasta \Dots} \\
\Prog{jellyfish merge} \Arg{hash1} \Arg{hash2} \Dots \\
\Prog{jellyfish dump}  \Arg{hash} \\
\Prog{jellyfish stats} \Arg{hash} \\
\Prog{jellyfish histo} \oOptArg{-h}{high} \oOptArg{-l}{low} \oOptArg{-i}{increment} \Arg{hash} \\
\Prog{jellyfish query} \Arg{hash}

\section{Description}

\Prog{Jellyfish} is a $k$-mer counter based on a multi-threaded hash
table implementation.

\subsection{Counting and merging}

To count $k$-mers, use a command like:

\begin{verbatim}
jellyfish count -m 22 -o output -c 3 -s 10000000 -t 32 input.fasta
\end{verbatim}

This will count the the 22-mers in input.fasta with 32 threads. The
counter field in the hash uses only 3 bits and the hash has at least
10 million entries.

The output files will be named output_0, output_1, etc. (the prefix is
specified with the \Opt{-o} switch). If the hash is large enough (has
specified by the \Opt{-s} switch) to fit all the $k$-mers, there will
be only one output file named output_0. If the hash filled up before
all the mers were read, the hash is dumped to disk, zeroed out and
reading in mers resumes.

To obtain correct results from the other sub-commands (such as histo,
stats, etc.), the multiple output files need to be merged into one
with the merge command. For example with the following command:

\begin{verbatim}
jellyfish merge -o output.jf output_*
\end{verbatim}

\subsection{Orientation}

When the orientation of the sequences in the input fasta file is not
known, e.g. in sequencing reads, using \LOpt{both-strands} (\Opt{-C})
makes the most sense.

For any $k$-mer $m$, its canonical representation is $m$ itself or its
reverse-complement, whichever comes first lexicographically. With the
option \Opt{-C}, only the canonical representation of the mers are
stored in the hash and the count value is the number of occurrences of
both the mer and its reverse-complement.


\subsection{Choosing the hash size}

To achieve the best performance, a minimum number of intermediary
files should be written to disk. So the parameter \Opt{-s} should be
chosen to fit as many $k$-mers as possible (ideally all of them) while
still fitting in memory.

For example, suppose we count $k$-mers in short sequencing reads with,
there is $n$ reads and there is an average of 1 error per reads where
each error generates $k$ unique mers. If the genome size is $G$, the
size of the hash to fit all $k$-mers at once would be: $(G +
k*n)/0.8$. The division by $0.8$ compensates for the maximum usage of
approximately $80\%$ of the hash table.

On the other hand, when counting $k$-mers in an assembled sequence of
length $G$, setting \Opt{-s} to $G$ is appropriate.

As a matter of convenience, Jellyfish understands ISO suffixes for the
size of the hash. Hence '-s 10M' stands 10 million entries while '-s
50G' stands for 50 billion entries.

The actual memory usage of the hash table can be computed as
follow. The actual size of the hash will be rounded up to the next
power of 2: $s=2^l$. The parameter $r$ is such that the maximum
reprobe value (\Opt{-p}) is less than $2^r$. Then the memory usage per
entry in the hash is (in bits, not bytes) $2k-l+r+1$. The total memory
usage of the hash table in bytes is: $2^l*(2k-l+r+1)/8$.

\subsection{Choosing the counting field size}
To save space, the hash table supports variable length counter, i.e. a
$k$-mer occurring only a few times will use a small counter, a $k$-mer
occurring many times will used multiple entries in the hash. The
\Opt{-c} specify the length of the small counter. The trade off is: a
low value will save space per entry in the hash but will increase the
number of entries used, hence maybe requiring a larger hash. In
practice, use a value for \Opt{-c} so that most of you $k$-mers
require only 1 entry. For example, to count $k$-mers in a genome,
where most of the sequence is unique, use \OptArg{-c}{1} or
\OptArg{-c}{2}. For sequencing reads, use a value for
\Opt{-c} large enough to counts up to twice the coverage.

\section{Options}
The following subcommand are used to look at the result: histo, dump, stats.
include(options.tex)
\section{Version}

Version: \Version\ of \Date

\section{Bugs}

\begin{itemize}
\item \Prog{jellyfish merge} has not been parallelized and is
  relatively slow.
\end{itemize}

\section{Copyright \& License}
\begin{description}
\item[Copyright] \copyright\ 2010, Guillaume Marcais \Email{guillaume@marcais.net} and Carl Kingsford \Email{carlk@umiacs.umd.edu}.

\item[License] This program is free software: you can redistribute it
  and/or modify it under the terms of the GNU General Public License
  as published by the Free Software Foundation, either version 3 of
  the License, or (at your option) any later version. \\
  This program is distributed in the hope that it will be useful, but
  WITHOUT ANY WARRANTY; without even the implied warranty of
  MERCHANTABILITY or FITNESS FOR A PARTICULAR PURPOSE.  See the GNU
  General Public License for more details. \\
  You should have received a copy of the GNU General Public License
  along with this program.  If not, see
  <\URL{http://www.gnu.org/licenses/}>.
\end{description}

\section{Authors}
\noindent
Guillaume Marcais \\
University of Maryland \\
\Email{gmarcais@umd.edu}

Carl Kingsford \\
University of Maryland \\
\Email{carlk@umiacs.umd.edu}

\LatexManEnd
\end{document}
